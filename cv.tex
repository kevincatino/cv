

\documentclass[letterpaper,11pt]{article}

\usepackage{latexsym}
\usepackage[empty]{fullpage}
\usepackage{titlesec}
\usepackage{marvosym}
\usepackage[usenames,dvipsnames]{color}
\usepackage{verbatim}
\usepackage{enumitem}
\usepackage[hidelinks]{hyperref}
\usepackage{fancyhdr}
\usepackage{tabularx}
\usepackage[utf8]{inputenc}
\usepackage[T1]{fontenc}
\usepackage[english,spanish]{babel}


% Define language-specific strings
\newcommand{\langstring}[2]{\iflanguage{#1}{#2}{}}
\newcommand{\en}[1]{\langstring{english}{#1}}
\newcommand{\es}[1]{\langstring{spanish}{#1}}



\pagestyle{fancy}
\fancyhf{} % clear all header and footer fields
\fancyfoot{}
\renewcommand{\headrulewidth}{0pt}
\renewcommand{\footrulewidth}{0pt}

% Adjust margins
\addtolength{\oddsidemargin}{-0.5in}
\addtolength{\evensidemargin}{-0.5in}
\addtolength{\textwidth}{1in}
\addtolength{\topmargin}{-.5in}
\addtolength{\textheight}{1.0in}

\urlstyle{same}

\raggedbottom
\raggedright
\setlength{\tabcolsep}{0in}

% Sections formatting
\titleformat{\section}{
  \vspace{-4pt}\scshape\raggedright\large
}{}{0em}{}[\color{black}\titlerule \vspace{-5pt}]

% Ensure that generate pdf is machine readable/ATS parsable
\pdfgentounicode=1

%-------------------------
% Custom commands
\newcommand{\resumeItem}[1]{
  \item\small{
    {#1 \vspace{-2pt}}
  }
}

\newcommand{\resumeSubheading}[4]{
  \vspace{-2pt}\item
    \begin{tabular*}{0.97\textwidth}[t]{l@{\extracolsep{\fill}}r}
      \textbf{#1} & #2 \\
      \textit{\small#3} & \textit{\small #4} \\
    \end{tabular*}\vspace{-7pt}
}

\newcommand{\resumeSubSubheading}[2]{
    \item
    \begin{tabular*}{0.97\textwidth}{l@{\extracolsep{\fill}}r}
      \textit{\small#1} & \textit{\small #2} \\
    \end{tabular*}\vspace{-7pt}
}

\newcommand{\resumeProjectHeading}[2]{
    \item
    \begin{tabular*}{0.97\textwidth}{l@{\extracolsep{\fill}}r}
      \small#1 & #2 \\
    \end{tabular*}\vspace{-7pt}
}

\newcommand{\resumeSubItem}[1]{\resumeItem{#1}\vspace{-4pt}}

\renewcommand\labelitemii{$\vcenter{\hbox{\tiny$\bullet$}}$}

\newcommand{\resumeSubHeadingListStart}{\begin{itemize}[leftmargin=0.15in, label={}]}
\newcommand{\resumeSubHeadingListEnd}{\end{itemize}}
\newcommand{\resumeItemListStart}{\begin{itemize}}
\newcommand{\resumeItemListEnd}{\end{itemize}\vspace{-5pt}}

%-------------------------------------------
%%%%%%  RESUME STARTS HERE  %%%%%%%%%%%%%%%%%%%%%%%%%%%%

\providecommand{\LANGUAGE}{english}

\begin{document}
\selectlanguage{\LANGUAGE}
%----------HEADING----------
% \begin{tabular*}{\textwidth}{l@{\extracolsep{\fill}}r}
%   \textbf{\href{http://sourabhbajaj.com/}{\Large Sourabh Bajaj}} & Email : \href{mailto:sourabh@sourabhbajaj.com}{sourabh@sourabhbajaj.com}\\
%   \href{http://sourabhbajaj.com/}{http://www.sourabhbajaj.com} & Mobile : +1-123-456-7890 \\
% \end{tabular*}

\begin{center}
    \textbf{\Huge \scshape Kevin Catino} \\ \vspace{1pt}
    \small +54 3489-55-1616 $|$ \href{mailto:x@x.com}{\underline{kevincatino18@gmail.com}} $|$ 
    \href{https://www.linkedin.com/in/khcatino}{\underline{www.linkedin.com/in/khcatino}} $|$
    \href{https://github.com/kevincatino}{\underline{github.com/kevincatino}}
\end{center}


%-----------EDUCATION-----------
\section{\en{Education}\es{Educación}}
  \resumeSubHeadingListStart
  \resumeSubheading
  {\en{Instituto Tecnológico de Buenos Aires}\es{Instituto Tecnológico de Buenos Aires}}{Buenos Aires, Buenos Aires}
  {\en{Software Engineering}\es{Ingeniería Informática} -- \en{Current GPA 8.85}\es{Promedio actual 8,85}}{\en{Mar}
  \es{Mar}. 2020 -- \en{Present (Currently in 5th year)}\es{Presente (Actualmente en 5to año)}}
    \resumeSubheading
      {Escuela Técnica Roberto Rocca}{Campana, Buenos Aires}
      {\en{Technical Degree in Electronics -- 9.02 GPA} \es{Tecnicatura en Electrónica -- Promedio 9,02}}{\en{Mar}
      \es{Mar}. 2013 -- \en{Dec}\es{Dic}. 2019}
  \resumeSubHeadingListEnd


%-----------LANGUAGES-----------
\section{\en{Languages}\es{Idiomas}}
 \begin{itemize}[leftmargin=0.15in, label={}]
    \small{\item{
     \textbf{\en{Spanish}\es{Español}}{: \en{Native}\es{Nativo}} \quad
     \textbf{\en{English}\es{Inglés}}{: \en{Proficient}\es{Competencia profesional}} \quad
     \textbf{\en{Japanese}\es{Japonés}}{: \en{Basic}\es{Básico}}
    }}
 \end{itemize}


%-----------EXPERIENCE-----------
\section{\en{Experience} \es{Experiencia}}
  \resumeSubHeadingListStart
  \resumeSubheading
  {\en{Full-stack developer (Back-end heavy)}\es{Desarrollador Full-Stack (enfocado en Back-end)}}{Buenos Aires, Buenos Aires}
  {Balanz Capital Valores}{\en{Oct.}\es{Oct.} 2022 -- \en{Present}\es{Presente}}
  \resumeItemListStart
  \resumeItem{\es{Desarrollo de sistema de generación de informes eficiente para la empresa, combinando tecnologías como
  Node.JS, SQL Server, HTML, CSS y módulos de terceros como Handlebars.JS y Puppeteer}
  \en{Developed a performant reporting system for the company combining technologies such as Node.JS, SQL Server, HTML,
CSS, and third party modules such as Handlebars.JS and Puppeteer}.}
  \resumeItem{\en{Contributed to the development of a Microservices architecture for client data processing in Golang}
  \es{Contribución en desarrollo de una arquitectura de Microservicios para el procesamiento de datos del cliente en Golang}.}
  \resumeItem{\en{Contributed to the development of libraries in Golang for multi-tenant storage management, error management and logging}
    \es{Contribución en el desarrollo de librerías en Golang para manejo de persistencia
  multi-tenant, manejo de errores y loggeo}.}
  \resumeItemListEnd
  \resumeSubheading
  {\en{Teaching Assistant for Object-Oriented Programming Course}\es{Asistente profesor para curso de Programación Orientada a Objetos}}
    {Buenos Aires, Buenos Aires}
  {\es{Instituto Tecnológico de Buenos Aires}\en{Buenos Aires Institute of Technology}}{\en{Aug}\es{Ago}. 2021 -- 
    \en{Dec}\es{Dic}. 2022}
 \resumeItemListStart
  \resumeItem{\en{Assisted in teaching the fundamentals of Object-Oriented Programming (OOP), covering topics like 
  polymorphism, inheritance, and composition}\es{Colaboración en la enseñanza de los fundamentos de la Programación 
  Orientada a Objetos (POO), abarcando temas como polimorfismo, herencia y composición}.}
  \resumeItem{\en{Supported the instruction of a Java-based course, which also introduced concepts like Iterators, 
  Interfaces, and Abstract classes}\es{Apoyo en la instrucción de un curso basado en Java, que también introdujo 
  conceptos como Iteradores, Interfaces y Clases abstractas}.}
  \resumeItem{\en{Introduced students to Ruby programming language, providing an overview of its syntax and core concepts}
  \es{Presentación a los estudiantes del lenguaje de programación Ruby, brindando una visión general de su sintaxis y conceptos fundamentales}.}

  \resumeItemListEnd
  \resumeSubheading
  {\es{Operador de aerosillas y soporte al cliente}\en{Customer Support and Lift Operator}}{Nagano, \en{Japan}\es{Japón}}
  {Shiga-kogen \en{Ski Resort}\es{Resort de Ski}}{\en{Dec}\es{Dic}. 2019 -- \en{Feb}\es{Feb}. 2020}
    % \resumeSubheading
    % {\en{Electrical Department Internship}\es{Pasantía en Departamento de Electricidad}}{Campana, Buenos Aires}
    %   {Termoeléctrica Manuel Belgrano}{\en{Sep}\es{Sep}. 2019 -- \en{Nov}\es{Nov}. 2019}

  \resumeSubHeadingListEnd


%
%-----------PROGRAMMING SKILLS-----------
\section{\en{Technical Skills}\es{Habilidades Técnicas}}
 \begin{itemize}[leftmargin=0.15in, label={}]
    \small{\item{
     \textbf{\en{Languages}\es{Lenguajes}}{: JavaScript, Java, C, Golang, Ruby, Terraform, Python, HTML, CSS} \\
     \textbf{\en{Databases}\es{Bases de datos}}{: PostgreSQL, SQL Server, MongoDB, Neo4j, Redis, DynamoDB} \\
     \textbf{Frameworks}{: React, Next.JS, TailwindCSS, Vue.JS, Vuetify} \\
     \textbf{\es{Otras herramientas de desarrollo/Plataformas}\en{Other Developer Tools/Platforms}}{: Git, Github, 
     Bitbucket, Gitlab, Neovim, Docker, Makefile, AWS, VS Code, IntelliJ, WebStorm, Android Studio, JIRA}
    }}
 \end{itemize}


%-----------PROJECTS-----------
\section{\en{Personal Projects}\es{Proyectos Personales}}
    \resumeSubHeadingListStart
      \resumeProjectHeading
          { \href{http://pawserver.it.itba.edu.ar/paw-2022b-6/}{\textbf{Unbiased}} $|$ \emph{Spring Web MVC, Java,
          JavaScript, Next.JS, Bootstrap, Postgres, Agile}}{\en{Aug}\es{Ago}. 2022 -- \en{Feb}\es{Feb}. 2023}
          \resumeItemListStart
          \resumeItem{\en{Full-stack web-app focused on the publishing and consumption of news articles from independent 
          journalists}\es{Desarrollo de una aplicación web full-stack enfocada en la publicación y consumo de artículos 
        de noticias de periodistas independientes}.}
            \resumeItem{\en{Helped develop a REST compliant backend API as well as the frontend with Next.JS}
            \es{Colaboración en el desarrollo de una API de backend compatible con REST, así como en el frontend con Next.JS}.}
            \resumeItem{\en{Integrated Hibernate to the backend persistence layer and migrated from JDBC}
            \es{Integración de Hibernate en la capa de persistencia del backend y migración desde JDBC}.}
          \resumeItemListEnd
      \resumeProjectHeading
          {\textbf{Cryptoriddles.io} $|$ \emph{JavaScript, Next.JS, TailwindCSS, Github, Agile}}{May 2022 -- Jul. 2022}
          \resumeItemListStart
          \resumeItem{\en{Educational group project which involved the full-stack development of a webpage}
          \es{Proyecto de grupo educativo que implicó el desarrollo full-stack de una página web}.}
            \resumeItem{\en{Developed the front-end from scratch with no previous React nor Next.JS experience}
            \es{Se desarrolló el frontend desde cero sin experiencia previa en React ni Next.JS}.}
          \resumeItemListEnd
      \resumeProjectHeading
          {\textbf{Homerun} $|$ \emph{JavaScript, Vue, Vuetify, Github, Java, Android Studio}}{May 2022 -- Jul. 2022}
          \resumeItemListStart
          \resumeItem{\en{Educational group project which involved the building of web and mobile interfaces for a home 
            automation mock-app from an exististing back-end API}\es{
Proyecto educativo de grupo que involucró la creación de interfaces web y móviles para una aplicación simulada de 
automatización del hogar a partir de una backend API existente}.}
            \resumeItem{\en{Developed the front-end with no previous Vue, Vuetify, JavaScript or Android development 
            experience}\es{Desarrollo de front-end sin experiencia previa en Vue, Vuetify, JavaScript o desarrollo Android}.}
          \resumeItemListEnd
    \resumeSubHeadingListEnd


  %-----------ACHIEVEMENTS-----------

\section{\en{Achievements}\es{Logros}}
\resumeSubHeadingListStart
\resumeSubheading
    {Cambridge Certificate in Advanced English}{Buenos Aires, Buenos Aires}
    {\en{Passed with 198/210}\es{Aprobado con 198/210} -- Upper C1 Level}{\en{Dec}\es{Dic}. 2019}
  \resumeSubheading
    {Techint "Becas al Mérito"}{Campana, Buenos Aires}
    {\en{Won Scholarship}\es{Beca recibida}}{\en{May}\es{Mayo} 2018}
  \resumeSubheading
    {\en{National Math Olympiads}\es{Olimpíada Matemática Argentina}}{Mar Del Plata, Buenos Aires}
    {\en{Passed Final Stage}\es{Certamen final aprobado}}{\en{Nov}\es{Nov}. 2017}
    % \resumeSubheading
    % {\en{National Math Team Olympiads}\es{Olimpíada Matemática Argentina} "Mateclubes"}{La Falda, Córdoba}
    % {\en{Passed Final Stage with special mention}\es{Instancia final aprobada con mención especial}}{2011, 2012, 2014}
\resumeSubHeadingListEnd



%-------------------------------------------
\end{document}
