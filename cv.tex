

\documentclass[letterpaper,11pt]{article}

\usepackage{latexsym}
\usepackage{titlesec}
\usepackage{marvosym}
\usepackage[usenames,dvipsnames]{xcolor}
\usepackage{verbatim}
\usepackage{enumitem}
\usepackage{fancyhdr}
\usepackage{tabularx}
\usepackage[utf8]{inputenc}
\usepackage[T1]{fontenc}
\usepackage[english,spanish]{babel}
\usepackage[letterpaper, left=0.5in, right=0.5in, top=0.5in, bottom=0.5in]{geometry}
\usepackage[hidelinks]{hyperref}


% Define language-specific strings
\newcommand{\en}[1]{\iflanguage{english}{#1}{}}
\newcommand{\es}[1]{\iflanguage{spanish}{#1}{}}



\pagestyle{fancy}
\fancyhf{} % clear all header and footer fields
\fancyfoot{}
\renewcommand{\headrulewidth}{0pt}
\renewcommand{\footrulewidth}{0pt}

\urlstyle{same}

\raggedbottom
\raggedright
\setlength{\tabcolsep}{0in}

% Sections formatting
\titleformat{\section}{
  \vspace{-4pt}\scshape\raggedright\large
}{}{0em}{}[\color{black}\titlerule \vspace{-5pt}]

% Ensure that generate pdf is machine readable/ATS parsable
\pdfgentounicode=1

%-------------------------
% Custom commands
\newcommand{\resumeItem}[1]{
  \item{\small #1}
}

\newcommand{\resumeSubheading}[4]{
  \item
    \begin{tabular*}{0.97\textwidth}[t]{l@{\extracolsep{\fill}}r}
      \textbf{#1} & #2 \\
      \textit{\small#3} & \textit{\small #4} \\
    \end{tabular*}
}

\newcommand{\resumeSubSubheading}[2]{
    \item
    \begin{tabular*}{0.97\textwidth}{l@{\extracolsep{\fill}}r}
      \textit{\small#1} & \textit{\small #2} \\
    \end{tabular*}
}

\newcommand{\resumeProjectHeading}[2]{
    \item
    \begin{tabular*}{0.97\textwidth}{l@{\extracolsep{\fill}}r}
      \small#1 & #2 \\
    \end{tabular*}
}

\newcommand{\resumeSubItem}[1]{\resumeItem{#1}}

\renewcommand\labelitemii{$\vcenter{\hbox{\tiny$\bullet$}}$}

\newcommand{\resumeSubHeadingListStart}{\begin{itemize}[leftmargin=0.15in, label={}]}
\newcommand{\resumeSubHeadingListEnd}{\end{itemize}}
\newcommand{\resumeItemListStart}{\begin{itemize}}
\newcommand{\resumeItemListEnd}{\end{itemize}\vspace{-5pt}}

%-------------------------------------------
%%%%%%  RESUME STARTS HERE  %%%%%%%%%%%%%%%%%%%%%%%%%%%%

\providecommand{\LANGUAGE}{english}

\begin{document}
\selectlanguage{\LANGUAGE}
%----------HEADING----------
\begin{center}
    \textbf{\Huge \scshape Kevin Catino} \\ \vspace{1pt}
    \small +54 3489-55-1616 $|$ \href{mailto:x@x.com}{\underline{kevincatino18@gmail.com}} $|$ 
    \href{https://www.linkedin.com/in/kevincatino}{\underline{www.linkedin.com/in/kevincatino}} $|$
    \href{https://github.com/kevincatino}{\underline{github.com/kevincatino}}
\end{center}


%-----------EDUCATION-----------
\section{\en{Education}\es{Educación}}
  \resumeSubHeadingListStart
  \resumeSubheading
  {\en{Instituto Tecnológico de Buenos Aires}\es{Instituto Tecnológico de Buenos Aires}}{Buenos Aires, Buenos Aires}
  {\en{Software Engineering}\es{Ingeniería Informática} -- \en{GPA 8.74}\es{Promedio 8,74}}{\en{Mar}
  \es{Mar.} 2020 -- \en{Jul. 2025}\es{Jul. 2025}}
  \resumeSubHeadingListEnd


%-----------LANGUAGES-----------
\section{\en{Languages}\es{Idiomas}}
 \resumeSubHeadingListStart
    \item \small{
     \textbf{\en{Spanish}\es{Español}}{: \en{Native}\es{Nativo}} \quad
     \textbf{\en{English}\es{Inglés}}{: \en{Proficient}\es{Competencia profesional}} \quad
     \textbf{\en{Japanese}\es{Japonés}}{: \en{Basic}\es{Básico}}
    }
 \resumeSubHeadingListEnd


%-----------EXPERIENCE-----------
\section{\en{Experience} \es{Experiencia}}
  \resumeSubHeadingListStart
  \resumeSubheading
  {\en{Cloud \& Platform Software Engineer (Semi Senior)}\es{Cloud \& Platform Software Engineer (Semi Senior)} -- Mercado Libre}{Buenos Aires, Buenos Aires}
  {(Kotlin, Java, AWS, GCP, Terraform, Docker, MySQL, Node.JS)}{\en{Oct.}\es{Oct.} 2025 -- \en{Present}\es{Presente}}
  % \resumeItemListStart
  % \resumeItem{\es{Diseño, desarrollo y mantenimiento de microservicios escalables y soluciones de plataforma en la nube utilizando Kotlin, Java y diversas tecnologías en la nube.}
  % \en{Designed, developed, and maintained scalable microservices and cloud platform solutions using Kotlin, Java, and various cloud technologies.}}
  % \resumeItem{\es{Gestión de infraestructura multi-cloud como código (IaC) con Terraform en AWS y GCP, garantizando implementaciones robustas y automatizadas.}
  % \en{Managed multi-cloud infrastructure as code (IaC) with Terraform across AWS and GCP, ensuring robust and automated deployments.}}
  % \resumeItemListEnd
  \resumeSubheading
  {\en{Software Engineer II}\es{Software Engineer II} -- Innovid}{Buenos Aires, Buenos Aires}
{(Scala, Java, Snowflake, Spark, Docker, Argo, AWS, Terraform, TeamCity)}
  {\en{June}\es{Junio} 2024 -- \en{Oct.}\es{Oct.} 2025}
  \resumeItemListStart
  \resumeItem{\es{Diseño, desarrollo y despliegue de proyecto en Scala+DBT+Spark para integración de datos de terceros a sistema de resolución de identidad, introduciendo Apache Iceberg como una alternativa de menor costo a Snowflake para almacenamiento.}
  \en{Designed, developed, and deployed a Scala+DBT+Spark project for integrating third-party data into an identity resolution system, introducing Apache Iceberg as a cost-saving storage alternative to Snowflake.}}
  \resumeItem{\es{Implementación y debuggeo de aplicaciones de Spark corriendo en AWS para procesamiento de billones de entradas por día.}
  \en{Implementated and debugged multiple Spark applications running on AWS processing billions of entries per day.}}
  \resumeItem{\es{Implementación de proyecto de DBT y workflow de k8s + Argo para procesamiento y envío de datos periódico a proveedores de identidad.}
  \en{Implemented a DBT project and a Kubernetes + Argo workflow for periodic data processing and delivery to identity providers.}}
  \resumeItem{\es{Mantenimiento y agregado de features varias a servicios existentes desarrollados en Scala, Python, Java, DBT y Node.js.}
  \en{Maintained and added various features to existing services developed in Scala, Python, Java, DBT, and Node.js.}}
  \resumeItemListEnd
  \resumeSubheading
  {\en{Software Developer}\es{Desarrollador de Software} -- Balanz Capital Valores}{Buenos Aires, Buenos Aires}
  {(Java, Golang, SQL Server, Postgres, Docker, Node.JS, AWS)}{\en{Oct.}\es{Oct.} 2022 -- \en{June}\es{Junio} 2024}
  \resumeItemListStart
  \resumeItem{\es{Constribución en el desarrollo de API con el uso de Spring Boot para permitir una comunicación con 
  Brokers externos con el objetivo de ampliar el servicio hacia el mercado de acciones de los Estados Unidos.}
  \en{Developed an API with Spring Boot to allow effective exchange of data with external brokers in order to provide 
  services to clients allowing operations with NYSE.}}
  \resumeItem{\es{Desarrollo de sistema de generación de informes eficiente para la empresa, combinando tecnologías como
-  Node.JS, SQL Server, HTML, CSS y módulos de terceros como Handlebars.JS y Puppeteer. Actualmente utilizado por miles 
de clientes en forma diaria.}
  \en{Developed a performant reporting system for the company combining technologies such as Node.JS, SQL Server, HTML,
CSS, and third party modules such as Handlebars.JS and Puppeteer. Currently being used on a daily basis by thousands of clients 
to visualize their portfolio holdings.}}
  \resumeItem{\en{Contributed to the development of a Microservices architecture for client data processing in Golang.}
  \es{Contribución en desarrollo de una arquitectura de Microservicios para el procesamiento de datos del cliente en Golang.}}
  \resumeItem{\en{Contributed to the development of libraries in Golang for multi-tenant storage management, error management and logging.}
    \es{Contribución en el desarrollo de librerías en Golang para manejo de persistencia
  multi-tenant, manejo de errores y loggeo.}}
  \resumeItemListEnd
  \resumeSubheading
  {\en{Teaching Assistant for OOP Course}\es{Asistente profesor para curso de OOP} -- ITBA}
    {Buenos Aires, Buenos Aires}
  {(Java, Ruby, UML)}{\en{Aug}\es{Ago}. 2021 -- 
    \en{Dec}\es{Dic}. 2022}
 \resumeItemListStart
  \resumeItem{\en{Assisted in teaching the fundamentals of Object-Oriented Programming (OOP), covering topics like 
  polymorphism, inheritance, and composition.}\es{Colaboración en la enseñanza de los fundamentos de la Programación 
  Orientada a Objetos (POO), abarcando temas como polimorfismo, herencia y composición.}}
  \resumeItem{\en{Supported the instruction of a Java-based course, which also introduced concepts like Iterators, 
  Interfaces, and Abstract classes.}\es{Apoyo en la instrucción de un curso basado en Java, que también introdujo 
  conceptos como Iteradores, Interfaces y Clases abstractas.}}
  \resumeItem{\en{Introduced students to Ruby programming language, providing an overview of its syntax and core concepts.}
  \es{Presentación a los estudiantes del lenguaje de programación Ruby, brindando una visión general de su sintaxis y conceptos fundamentales.}}

  \resumeItemListEnd
  \resumeSubheading
  {\es{Operador de aerosillas y soporte al cliente}\en{Customer Support and Lift Operator}}{Nagano, \en{Japan}\es{Japón}}
  {Shiga-kogen \en{Ski Resort}\es{Resort de Ski}}{\en{Dec}\es{Dic}. 2019 -- \en{Feb}\es{Feb}. 2020}
    \resumeSubheading
    {\en{Electrical Department Internship}\es{Pasantía en Departamento de Electricidad}}{Campana, Buenos Aires}
    {Termoeléctrica Manuel Belgrano}{\en{Sep}\es{Sep}. 2019 -- \en{Nov}\es{Nov}. 2019}

  \resumeSubHeadingListEnd


%
%-----------PROGRAMMING SKILLS-----------
\section{\en{Technical Skills}\es{Habilidades Técnicas}}
 \begin{itemize}[leftmargin=0.15in, label={}, itemsep=-4pt]
    \resumeItem{\textbf{\en{Languages}\es{Lenguajes}}: Java, Javascript, Scala, C, Golang, Ruby, HCL, Python, HTML, CSS}
    \resumeItem{\textbf{\en{Databases}\es{Bases de datos}}: PostgreSQL, SQL Server, Snowflake, MongoDB, Neo4j, Redis, DynamoDB}
    \resumeItem{\textbf{Frameworks}: Spring, Express.JS, React, Next.JS, TailwindCSS, Vue.JS, Vuetify}
    \resumeItem{\textbf{\es{Otras herramientas de desarrollo/Plataformas}\en{Other Developer Tools/Platforms}}: Kubernetes, Argo, Terraform, Git, Github, Bitbucket, Gitlab, Neovim, Docker, TeamCity, Gitlab CI, Makefile, AWS, VS Code, IntelliJ, WebStorm, Android Studio, JIRA}
    \resumeItem{\textbf{\en{AI Tools}\es{Herramientas de IA}}: Cursor, Codex CLI, Claude Code, Notebook LM, Gemini CLI, Gemini Studio, Chat GPT}
 \end{itemize}


%-------------------------------------------
\end{document}
